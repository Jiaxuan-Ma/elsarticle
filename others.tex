

\documentclass[preprint,review,12pt,authoryear]{elsarticle}

\usepackage{amssymb}
%% The amsthm package provides extended theorem environments
\usepackage{amsthm}
\usepackage{amsmath}
\usepackage{titlesec} % 用于自定义章节标题样式
\usepackage{caption}
\usepackage{tabularx}
\usepackage{booktabs}
\usepackage{hyperref}

%% The lineno packages adds line numbers. Start line numbering with
%% \begin{linenumbers}, end it with \end{linenumbers}. Or switch it on
%% for the whole article with \linenumbers.
%% \usepackage{lineno}



\begin{document}
% \captionsetup[table]{
%   name={Table S},
%   labelformat=simple, % 使用简单的编号格式
%   labelsep=space,     % 编号和标题之间的分隔符为一个空格
%   justification=centering, % 居中对齐
% }


\begin{center}
\Large 
Motivation
\end{center}

\begin{center}
    Jiaxuan Ma ~~~~~~~~~~~ Email:mjx@shu.edu.cn
\end{center}

I am deeply motivated to contribute to this groundbreaking project that aims to revolutionize forging process optimization through AI-driven solutions. Below, I outline my key motivations for joining this initiative:

Creating a tool that can simulate forging processes in milliseconds, rather than hours, aligns perfectly with my research in accelerating simulation shape morphing of smart materials under mechanical-electrical coupling condition. I have developed U-LSTM and U-ViT models that can accelerate these simulations by four orders of magnitude while maintaining the high precision of FEM. Additionally, I introduced a data-based control framework for solving smart actuator underwater locomotion problems using deep reinforcement learning (DRL).

I also developed a visualization AI platform named \href{https://github.com/Jiaxuan-Ma/MLMD}{MLMD} for material performance prediction and design. This platform includes data analysis, property prediction, transfer learning, surrogate optimization and active learning. It offers a web-based, user-friendly interface that requires no programming and can be accessed anywhere, anytime. Furthermore, I have researched mechanical meta-material design through implicit-based functions and developed TPMS-GAN for real-time, multi-objective design, considering the costs associated with each objective.

In summary, I am inspired by the opportunity to apply my skills and knowledge to a project that has the potential to transform the forging industry. This project aligns with my career aspirations and personal values, and I am excited to be part of a team that is shaping the future of manufacturing.




%% If you have bibdatabase file and want bibtex to generate the
%% bibitems, please use
%%
% \bibliographystyle{elsarticle-harv} 
% \bibliography{refs}

%% else use the following coding to input the bibitems directly in the
%% TeX file.

% \begin{thebibliography}{00}

% %% \bibitem[Author(year)]{label}
% %% Text of bibliographic item

% \bibitem[ ()]{}

% \end{thebibliography}
\end{document}

\endinput
%%
%% End of file `elsarticle-template-harv.tex'.

